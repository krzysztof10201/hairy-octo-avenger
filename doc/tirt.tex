\documentclass[a4paper,12pt]{article}
\usepackage[margin=1in]{geometry}
\usepackage[utf8]{inputenc}
\usepackage[polish]{babel}
\usepackage{polski}

\usepackage[hidelinks]{hyperref}
\usepackage{graphicx}

\linespread{1.2}

\begin{document}

\tableofcontents

\vspace{0.5cm}
\hrule

\section{Metody porównywania wyników}

Potrzebne są metody, które pozwolą porównać dane rzeczywiste z modelowymi i stwierdzić, w jakim stopniu są do siebie podobne.

\begin{itemize}
    \item Podział obszaru symulacji na podobszary i zliczanie agentów na każdym z nich. Następnie porównywanie wyników różnych symulacji na danym podobszarze. Może istnieje także sposób na uwzględnienie sąsiadów, lub inny sposób zwiększenia elastyczności porównania.

    \item Porównywanie wartości liczbowych (średnia, minimum, maksimum, etc.), które mówią o rozmaitych parametrach ruchu, np.:
    \begin{itemize}
        \item szybkość,
        \item przyspieszenie,
        \item zmiana kierunku,
        \item przestój,
        \item droga.
    \end{itemize}

    \item Zasięg sieci bezprzewodowej, czyli w uproszczeniu może to być np. odległość od wyznaczonych z góry punktów. Bardziej wymyślne sposoby dokonywania pomiarów prawdopodobnie zaczną nachodzić na zagadnienie przydziału stacji bazowych.

    \item Preferencje odwiedzania pewnych obszarów, częstotliwość powrotu do wcześniej odwiedzonych lokacji.

    \item Częstotliwość spotkań z innymi agentami (lub jakieś inne interakcje agentów?).

    \item Wykresy trajektorii.
\end{itemize}

\pagebreak

\section{Modele mobilności}

\subsection{Modele bez ograniczeń geograficznych}

\begin{enumerate}
    \item Modele losowe
    \begin{itemize}
        \item random walk
        \item \textbf{random waypoint}
        \item random direction model (?)
    \end{itemize}
    \item Modele z zależnościami czasowymi
    \begin{itemize}
        \item \textbf{Gauss-Markov}, który przy odpowiednim doborze parametrów sprowadza się także do innych modeli:
        \begin{itemize}
            \item random walk
            \item fluid flow
        \end{itemize}
        \item smooth random
    \end{itemize}
    \item Modele z zależnościami przestrzennymi
    \begin{itemize}
        \item \textbf{reference point group}
        \item column (chyba trochę nudny)
        \item pursue
        \item nomadic community
    \end{itemize}
\end{enumerate}

\subsection{Modele z ograniczeniami geograficznymi}

\begin{itemize}
    \item \textbf{pathway} (można łatwo połączyć z modelami bez ograniczeń geograficznych, można też uznać ścieżki za zaledwie wskazówki dotyczące poruszania się)
    \item obstacle (jeśli uznać przeszkody za ograniczenia twarde, to może być problem z porównywaniem do innych modeli; można też próbować zamienić przeszkody na ograniczenia miękkie, np. poprzez spowolnienie ruchu pokonujących je agentów)
\end{itemize}

\section{Projekt aplikacji}

\subsection{Diagram klas}

\includegraphics[width=\linewidth]{class.pdf}

\end{document}
